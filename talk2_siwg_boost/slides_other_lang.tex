% Created 2011-09-16 Fri 15:30
\documentclass{beamer}
\usepackage[utf8]{inputenc}
\usepackage[T1]{fontenc}
\usepackage{fixltx2e}
\usepackage{graphicx}
\usepackage{longtable}
\usepackage{float}
\usepackage{wrapfig}
\usepackage{soul}
\usepackage{textcomp}
\usepackage{marvosym}
\usepackage{wasysym}
\usepackage{latexsym}
\usepackage{amssymb}
\usepackage{hyperref}
\tolerance=1000
\mode<beamer>{\usetheme{Madrid}}
\documentclass[8pt]{beamer} 
\providecommand{\alert}[1]{\textbf{#1}}

\title{Using Python With Other languages (Boost and SWIG)}
\author{Siddhant Sanyam}
\date{2011-09-07 Wed}

\begin{document}

\maketitle

\begin{frame}
\frametitle{Outline}
\setcounter{tocdepth}{3}
\tableofcontents
\end{frame}



\section{Python With C++}
\label{sec-1}
\begin{frame}
\frametitle{Why C++ ?}
\label{sec-1_1}


\begin{itemize}
\item Because it is as good as C (well, most of the time)
\item Extending C API is ugly, and terribly confusing
\item Because it is Object Oriented
\begin{itemize}
\item Thus you can think in OOP
\item Resembles Python (the OO part)
\end{itemize}
\item It is more powerfull as a language
\item Offers competitive performance as C
\item Because of Boost (wait till I introduce you to it)
\end{itemize}
\end{frame}
\begin{frame}
\frametitle{What can be done with Python and C++ ?}
\label{sec-1_2}


\begin{itemize}
\item Same stuff you can do with C but more
\item You can have class abstractions
\item Extending Python with C++
\item Embedding Python in C++
\item Python is more close to C++ than to C
\end{itemize}
\end{frame}
\begin{frame}
\frametitle{Available options to extend Python with C++}
\label{sec-1_3}

   There are multiple options that are available to extend Python with C++

\begin{itemize}
\item Using the ever-friendly Python C API
\item Boost's API for C++
\item SWIG
     Let's take each one of them one by one
\end{itemize}
\end{frame}
\begin{frame}
\frametitle{Extending Python with C++ Using Python C API}
\label{sec-1_4}


\begin{itemize}
\item Basic idea is that since C++ is essential ``C with Classes'', we can use C++ as C and use the Python C API
\item But that sucks! \ldots{} at multiple levels
\begin{itemize}
\item You have to deal with the ugly API and you miss the comfort of Boost and SWIG
\item You are using C++ as C, so no classes and objects, why not use C then?
\item For simple cases when you want to enjoy C++ as a language (with STL and all), it makes sense
\item But you can't expose the class-like interface of C++
\end{itemize}
\item When will you use it?
     Almost never. Why? Because there are superior alternative available
\end{itemize}
\end{frame}
\section{Extending C++ using Boost.Python}
\label{sec-2}
\begin{frame}
\frametitle{Introduction to Boost.Python}
\label{sec-2_1}

  So basically, it is a wrapper class on the Python C API
  It's awesome

\begin{itemize}
\item Because it is easy to use (majorly because of the syntax)
\item You can expose the class interface of C++ with this
\item Great for small to medium sized projects
\item It does operator overloading too!
\item Inheritance is also supported
\item In short, it brings C++ almost completely to Python
\end{itemize}
\end{frame}
\begin{frame}
\frametitle{Boost.Python kickstart}
\label{sec-2_2}

   We assume that you have Boost.Python installed on your system.

\begin{itemize}
\item Boost.Python also works without installing it, although it is recommended that you do install it.
\item Boost.Python requires you to include a simple header file <boost/python.hpp>
\item We'll be basically writing Python modules using C++
\end{itemize}
\end{frame}
\begin{frame}
\frametitle{Compiling and Build process}
\label{sec-2_3}

   You have many options to build and compile Boost

\begin{itemize}
\item bjam, the recommended way of building Boost
\item mannually linking and compiling using g++
\end{itemize}
   We'll assume g++ to be de facto standard compiler. Python 2.4 and above is recommended.
   
\end{frame}
\begin{frame}
\frametitle{Hello Boost!}
\label{sec-2_4}

   Our purpose is to write a module \texttt{hello} which has a function called \texttt{say}

\begin{itemize}
\item We'll writing the definiation of the function \texttt{say} in C++
\item Then we'll use Boost.Python to expose the function in a module called \texttt{hello}
\item It's similar stuff we did with Python C API, but I think you'll appreciate the neatness.
\end{itemize}
\end{frame}
\begin{frame}[fragile]
\frametitle{Here is the code for \texttt{say}}
\label{sec-2_5}

   \texttt{say} takes one \texttt{unsigned int} as an arguement and returns `Hello', `Boost.Python' or 
   `World' depending on if the \texttt{int} passed was 0, 1 or 2.
   Definition of \texttt{say}
\begin{verbatim}
1:  char const* say(unsigned x)
2:  {
3:    static char const* const msgs[] = { "Hello", "Boost.Python", "World!" };
4:  
5:    if (x > 2) 
6:      throw std::range_error("say: index out of range");
7:  
8:    return msgs[x];
9:  }
\end{verbatim}

\begin{itemize}
\item Simple enough right?
\item Well, writing C++ is always the simple part, the important part is making a module out of it.
\end{itemize}
\end{frame}
\begin{frame}[fragile]
\frametitle{The Boost's Magic}
\label{sec-2_6}

   Tuns out that to write the module \texttt{hello} and putting function \texttt{say} in it is pretty easy in Boost.Python
\begin{verbatim}
1:  #include <boost/python.hpp>
2:  using namespace boost::python;
3:  BOOST_PYTHON_MODULE(hello)
4:  {
5:    def("say", //the name of the function to be exposed in Python
6:        say,   //the function name in c++
7:        "return one of 3 part of Hello Boost.Python World"); //a small description of function for the help function
8:  }
\end{verbatim}
   That's it!

\begin{itemize}
\item Now you know what is meant by ``easy to use''
\end{itemize}
\end{frame}
\begin{frame}[fragile]
\frametitle{Compile and Go}
\label{sec-2_7}

   As I said there are multiple way to compile and go. Easiest is through bjam
   Just execute the command
\begin{verbatim}
    bjam --tools-set
\end{verbatim}

   
\end{frame}
\begin{frame}
\frametitle{Now that's cool, but what about higher stuff?}
\label{sec-2_8}


\begin{itemize}
\item We'll take each example file one by one
\item Please use a text editor with Line numbering because we'll be using the line numbers to reference examples
\end{itemize}
\end{frame}
\begin{frame}
\frametitle{Examples}
\label{sec-2_9}


\begin{itemize}
\item Exposing Classes
\item Operator Overloading
\item Inheritance
\item Virtual Function
\item Serialization
\item Object coversion
\end{itemize}
\end{frame}
\section{Extending Python with C and C++ using SWIG}
\label{sec-3}
\begin{frame}
\frametitle{SWIG}
\label{sec-3_1}

   It's a wrapper and interface writer for C/C++
   It means

\begin{itemize}
\item You write C or C++ code, and this tool will write the interface and wrapper to other languages for your code.
\item It supports many languages like Python, Tcl, Ruby, Guile, and Java
\item It supports nearly every feature of C++
\item Usually it offers a neat distinction betwen the original code and the interface (thus you don't have to modify your orignal sources much)
\end{itemize}
\end{frame}
\begin{frame}
\frametitle{Advantage of SWIG}
\label{sec-3_2}


\begin{itemize}
\item It's a uniform over different language
\item Hence it is awesome way to have wrappers for your C++ library.
\item It's much more complete to cover features of C++
\item If your interface is in separate file, writing SWIG interface is trivial.
\item It requires none or very less change in your original source code.
\end{itemize}
\end{frame}
\begin{frame}
\frametitle{How does it work for Python}
\label{sec-3_3}


\begin{itemize}
\item You first write your C/C++ interface and definition
\item You write a dot i file (example.i) which specifies the interface of the wrapper
\item Use SWIG to generate wrapper files of Python
\item Compiler C/C++ code to a shared library
\item Use your freshly baked module.
\end{itemize}
\end{frame}
\begin{frame}
\frametitle{SWIG kickstart}
\label{sec-3_4}


\begin{itemize}
\item You should be having SWIG installed, Python installed.
\item You optionally would like to have a c++ compiler installed :P
\item We'll be using our beloved GNU's GCC in examples.
\end{itemize}
\end{frame}
\begin{frame}[fragile]
\frametitle{Hello SWIG}
\label{sec-3_5}

   First create the C part
   Header file/Interface file

\begin{verbatim}
1:  /*hello.h*/
2:  
3:  /*Signature, Prototype, Definition of the function*/
4:  int factorial(int n);
\end{verbatim}
   
\end{frame}
\begin{frame}[fragile]
\frametitle{The definition}
\label{sec-3_6}

   Now we'll write the definition of our function
\begin{verbatim}
 1:  /* hello.c */
 2:  /* Definition of factorial */
 3:  #include "hello.h"
 4:  int factorial(int n) {
 5:    if (n < 0){ 
 6:      return 0; // Hmm, we're treating this error case
 7:    }
 8:    if (n == 0) {
 9:      return 1;
10:    }
11:    else {
12:      return n * factorial(n-1);
13:    }
14:  }
\end{verbatim}
\end{frame}
\begin{frame}[fragile]
\frametitle{Using in a C program}
\label{sec-3_7}

   This is how the main file will use this library in your  typical C program
\begin{verbatim}
1:  #include <stdio.h>
2:  #include "hello.h"
3:  
4:  int main(void)
5:  {
6:    int a = 5;
7:    printf("%i", factorial(a) );
8:  }
\end{verbatim}
   And then you compile it with hello.c
\begin{verbatim}
    gcc main.c hello.c -o hello_executable
    ./hello_executable
\end{verbatim}
\end{frame}
\begin{frame}
\frametitle{This was C, let's make it Pythonable}
\label{sec-3_8}


\begin{itemize}
\item You will appreciate the simplicity of SWIG.
\item Our C interface is already done.
\item We'll just write a SWIG's .i file to expose the interface to SWIG
\end{itemize}
\end{frame}
\begin{frame}[fragile]
\frametitle{hello.i}
\label{sec-3_9}

\begin{verbatim}
/* File: hello.i */
%module hello
 /*
   The following blocks are included as is in the wrapper file
   Important to include the interface file in wrapper file
   because the input file is just used for generating the wrapper code
 */

%{
#include "hello.h"
%}
/*
  We'll now declare the interface for the wrapper file.
  At worst case, we can simply #include "hello.h" again
*/
int factorial(int);
\end{verbatim}
     That's it.
\end{frame}
\begin{frame}[fragile]
\frametitle{Let it SWIG}
\label{sec-3_10}

   Now will run SWIG on this input file we just wrote.
\begin{verbatim}
    swig -python hello.i
\end{verbatim}

   Now you should have these files
\begin{verbatim}
    hello.py hello.h hello.c hello_wrap.c
\end{verbatim}

   Compile your stuff
\begin{verbatim}
    gcc -c hello.c hello_wrap.c -I /usr/include/python2.7 -I/usr/lib/python2.7/
\end{verbatim}

   You'll be having \texttt{hello.o} and \texttt{hello\_wrap.o} . It's time to link them into a shared library
   The name of the so file is module name prefixed by an underscore
\begin{verbatim}
    ld -shared -o _hello.so hello.o hello_wrap.o
\end{verbatim}

   Ta da! we have got our \texttt{\_hello.so} ready.
\end{frame}
\begin{frame}
\frametitle{Lock and load}
\label{sec-3_11}


\begin{itemize}
\item Now we have \texttt{hello.py} (generated by SWIG using hello.i)
\item We can simply \texttt{import hello} to load the module.
\item The loading of the .so file is job of hello.py
\end{itemize}
\end{frame}
\begin{frame}[fragile]
\frametitle{Using Module}
\label{sec-3_12}

\begin{verbatim}
python
>>> import hello
>>> hello.factorial(5)
120
>>>
\end{verbatim}

\begin{itemize}
\item Enjoyed it! Let's do it for C++ now
\end{itemize}
\end{frame}
\begin{frame}
\frametitle{First thing first, writing the C++ code}
\label{sec-3_13}


\begin{itemize}
\item We could all the above example of simple function exposing in C++ too
\item But since it is C++, We'll  do OOP by having a simple class of Point number
\end{itemize}
\end{frame}
\begin{frame}[fragile]
\frametitle{The interface file}
\label{sec-3_14}

\begin{verbatim}
/* File : Point.h */

class Point
{
private:
  int x;
  int y;
public:
  Point(int,int);
  float dist_origin(); //distance from origin
};
\end{verbatim}
\end{frame}
\begin{frame}[fragile]
\frametitle{Implementation file for Point}
\label{sec-3_15}

\begin{verbatim}
/* File: Point.cpp */
#include "Point.h"
#include <cmath>
Point::Point(int X, int Y)
{
  x=X;
  y=Y;
}
float Point::dist_origin()
{
  return pow(x*x + y*y, 0.5);
}
\end{verbatim}
\end{frame}
\begin{frame}[fragile]
\frametitle{Using this Library in C++ (normally)}
\label{sec-3_16}

\begin{verbatim}
/* File: main.cpp */
#include <iostream>
#include "Point.h"
int main()
{
  Point p1= Point(2,3);
  std::cout << p1.dist_origin();
}
\end{verbatim}
   Let's compile this, like we do it in C++
\begin{verbatim}
    g++ main.cpp Point.cpp -o point_example
    ./point_example
\end{verbatim}

   
\end{frame}
\begin{frame}[fragile]
\frametitle{Now time for it to be eaten by Python}
\label{sec-3_17}

   We'll write the SWIG's .i file
\begin{verbatim}
/* File : Point.i */

%module Point // Module name

%{
#include "Point.h" //including it for  the Point_wrap.cxx file
  %}

/*
  We are being lazy over here. Instead of putting the
  actual interface, we just copied it from Point.h
*/
%include "Point.h"
\end{verbatim}
   Let's quickly make the wrapper c++ file and the interface .py files using SWIG
\begin{verbatim}
    swig -python -c++ Point.i
\end{verbatim}

   Notice that we used \texttt{-c++} to tell SWIG that we'll be using Classes etc.
\end{frame}
\begin{frame}[fragile]
\frametitle{Compiling the resultant files}
\label{sec-3_18}

   The following file are generated
\begin{verbatim}
    Point_wrap.c Point_wrap.cxx Point.py
\end{verbatim}

   Let's use g++ to compile our wrapper and source file
\begin{verbatim}
    g++ -O2 -fPIC -c Point.cpp Point_wrap.cxx -I /usr/include/python2.7/
\end{verbatim}

   Now we'll make the shared library \texttt{\_Point.so} (note the naming convention is same)
\begin{verbatim}
    g++ -shared -o _Point.so Point.o Point_wrap.o 
\end{verbatim}

   That's it, we can start using this class in our Python code
\end{frame}
\begin{frame}[fragile]
\frametitle{Run the module}
\label{sec-3_19}

\begin{verbatim}
python
>>> import Point
>>> p = Point.Point(3,4)
>>> p.dist_origin()
5.0
\end{verbatim}
   Wasn't that simple?
\end{frame}
\begin{frame}
\frametitle{SWIG Examples}
\label{sec-3_20}


\begin{itemize}
\item Global Variables
\item SWIG Directives
\item Constants
\item Pointers
\item Operator Overloading
\item Inheritance
\end{itemize}



   
   
\end{frame}

\end{document}
